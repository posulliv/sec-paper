\section{Related Work}
\label{sec:related}

Our work is related to many techniques that attempt to defend against
attacks on vulnerabilities in applications. In this section, we
elaborate on some of the pieces of work most closely related to
ours. We present the various attack techniques utilized by
attackers that are relevant for this research and then we go on to
present various techniques proposed for mitigating these attack
techniques. We also briefly discuss related work in
binary rewriting.

\subsection{Catalog of Attack Techniques}

\mypar{Buffer Overflow Attacks} A buffer overflow refers to a
situation that can occur when code writes into a bounded array, or
buffer, and the writes are not correctly guarded against
overflow. Data copied into the buffer whose length is larger than the
buffer's size is referred to as a buffer overflow. Writes into a
buffer that are not correctly guarded may overwrite and corrupt a
variety of vulnerable locations that may also be stored nearby the
buffer, including return addresses, base pointers, function pointers,
and \emph{longjmp} buffers. Although buffer overflows have
historically most often occurred on the stack, they are also possible
on heap and global segment buffers.  For example a global buffer's
overflow may overwrite a function pointer or \emph{longjmp} buffer
also in the global segment.

Buffer overflow attacks work by changing the value of the code pointer
stored in vulnerable locations such as return addresses, function
pointers and \emph{longjmp} buffers. The code pointer is overwritten
by a new value pointing to code of the attacker's choice.  Base
pointers are also vulnerable even though they are not code since they
can be used for attacks \cite{wilander2003}. A return address attack
was first described in detail by AlephOne in 1996
\cite{one1996}. However, attacks of this kind date back to before 1988
when the technique was used in the \begin{em}fingerd\end{em} exploit
  of the Morris worm.

Commonly, an attacker would choose their input data so that the
machine code for an attack payload would be present at the modified
return address. When the vulnerable function returns, and execution of
the attack payload begins, the attacker has gained control of the
behavior of the target software. The attack payload is often called
shellcode, since a common goal of an attacker is to launch a command
line interpreter (referred to as a shell in UNIX like environments)
under their control.

\mypar{Return-to-libc Attacks} As an alternative to supplying
executable code (referred to as direct code injection), an attacker
might be able to craft an attack that executes existing machine code
(indirect code injection). This class of attacks has been referred to
as jump-to-libc or return-to-libc (arc injection
\cite{cowan2000buffer} has also been used to refer to this class of
attacks) because the attack often involves directing execution towards
machine code in the standard C library (libc)
\cite{cowan2000buffer}. The standard C library is often the target for
attacks of this type since it is loaded in nearly every UNIX program
and it contains routines of the sort that are useful for an
attacker. This technique was first suggested by Solar Designer in 1997
\cite{designer1997return}. Attacks of this kind can evade defense
mechanisms that protect the stack such as stack canaries and it is
also effective against defenses that only allow memory to be writable
or executable.

Traditionally, attacks of this kind have targeted the system function
in the standard system library which allows the execution of an
arbitrary command with arguments. In this case, the return address of
a vulnerable function would be modified to point to the address of the
system function which would then be executed with attacker supplied
arguments. Since the system function executes a command on a system,
if an attacker can control the arguments to this function, they could
execute an arbitrary command on the system under attack. However,
recent attacks have been demonstrated which do not depend on calling
functions in the standard C library.

% Padraig: Please explain the exact mechanism of how the above
% return-to-libc attack works, for example the system function attack
% you mention above.  As this subsection is written now, I cannot
% understand how any return-to-libc attack works.  DONE

\subsection{Catalog of Defense Techniques}

\mypar{Compile Time Defenses} StackGuard \cite{stackguard-98} places a
'canary' (a memory location) on the stack between local variables and
the return address. This canary value is designed to warn of stack
corruption since validating the integrity of the canary value is an
effective means of ensuring that the function return address has not
been corrupted. Microsoft's compiler also supports the insertion of
stack canaries with the /GS option.

ProPolice \cite{eto2002propolice} is similar to StackGuard in that it
places a canary value on the stack. However, ProPolice also changes
the stack layout to place arrays and other function-local buffers
above all other function-local variables. Copies of all function
arguments are also made into new, function-local variables that also
sit below any buffers in the function. As a result, these variables
and arguments are not subject to corruption through an overflow of
these buffers.

PointGuard \cite{pointguard} protects all code pointers within a
program. The defense consists of encrypting pointer values in memory
and only decrypting the pointers when they are loaded into CPU
registers. The encryption key used is a randomly generated during
process creation and is thus unknown to an attacker. Without knowledge
of the encryption key, an attacker can not modify any value stored in
memory. As a result, pointer values are not subject to corruption.

StackGuard, PointGuard, and ProPolice involve compile-time analysis
and transformation. Thus, unless the source code for an application is
available, these techniques can not be used thereby hindering the
ability to easily deploy these techniques. In practice only the
developer can use these defenses, and only if the compiler his or her
organization uses supports it. Our techniques do not suffer from this
drawback since they can be easily deployed on any binary produced from
any source language and compiler, by not only the developer, but the
end-user as well.

% Padraig: the above paragraph references Pointguard, but it is not
% explained anywhere before that point.  Please fix.  DONE

\mypar{Instruction Set Randomization} Instruction-set randomization
\cite{isr} is a technique for protecting against buffer
overflows (and many kinds of code injection attacks). This approach
randomizes the underlying system's instructions so that foreign code
injected by an attacker would fail to execute correctly since the
attacker does not know the instruction set of the target
system. However, as mentioned by the authors in \cite{isr}, the main
drawback of this technique as applied to binary code that it needs
specialized hardware support in the processor. Thus, even though
instruction-set randomization offers a strong defense against buffer
overflow attacks the fact that unless it is supported by specialized
hardware, it incurs significant overheads means that it is unlikely to
see adoption in practice for the foreseeable future.

% Padraig: In dont understand the last two sentences.  If interpretive
% languages are not vulnerable, then how can the defense be effective?
% You can replace "are not vulnerable to" by "have not seen many".
% (or delete the two sentences) DONE - I removed the 2 sentences

Strata (a dynamic binary translation framework) and Diablo (a
link-time binary rewriter) were used to implement instruction set
randomization \cite{strata}. Diablo is used to prepare a binary for
string encryption and introduce the information necessary to detect
foreign code. Strata is then used to provide the necessary virtual
execution environment for safe execution. The main contribution of
this work is that the instruction-set randomization implementation is
efficient while requiring no special hardware support. However, the
runtime overheads reported are still high because of the necessary
software ISA translation at run-time, and the inherent overheads of a
dynamic translator. These run-time overheads and likely to limit the
practical adoption of such a system. Moreover any user of a dynamic binary rewriter must install it in addition to the application desired, making it inconvenient to use.

A static (off-line) binary rewriter such as the one used for our
research suffers from none of these issues. No special hardware is
required in order to utilize a binary rewriter and overheads are
relatively low since no ISA translation is done, and since no dynamic
translator is used, no additional software in addition to the application is needed for execution. In our system, if an original binary was compiled
without optimizations, we often see a significant run-time improvement
when rewritten.

\mypar{Address Space Layout Randomization} Address Space Layout
Randomization (ASLR) can be seen as a relatively coarse-grained form
of software diversity \cite{aslr}. ASLR shuffles, or randomizes, the
layout of software in the memory address space. The common
implementation of this scheme is at the OS level. Thus, when a process
is launched the address space layout of the process will be different
from a previous invocation of the same process. It is effective at
preventing remote attackers that have no existing means of running
code on a target system from crafting attacks that depend on
addresses. ASLR is not intended to defend against attackers that are
able to control the execution of a piece of software; it is mainly
intended to hamper remote attackers from attempting to use the same
attack repeatedly. Finally, its utility on 32-bit architectures is
limited by the number of bits available for address randomization
\cite{aslr32}.

% Padraig: (1) Who implements existing ASLR schemes -- a compiler,
% hardware, binary rewriter or OS?  Please clarify in above.  (2) Also
% you write "ASLR is not intended to defend against attackers that are
% able to control the software execution".  What does this mean?
% (Please rewrite to clarify.)  (3) I dont understand the scheme --
% does ASLR change the binary for each target system or once at
% compile-time?  If it is once then the randomized binary can be
% studied by the attacker and then attacked easily, is it not?  The
% basic problem is that the above paragraph does not clearly describe
% precisely what entity does ASLR or how it works and neither does it
% give an example of it.  Please rewrite to clarify all of these
% points.
%
% DONE - I attempted to address this. ASLR is not compile time, the
% address space is randomized each time a process is launched through
% a patch the loader in the operating system.


A binary rewriter could easily be used to provide a similar defense
mechanism as ASLR. An interesting future avenue of research is to
investigate software diversity through binary rewriting.

\mypar{Control Flow Integrity} Control Flow Integrity (CFI)
\cite{cfi-05} is a basic safety property that can prevent attacks from
arbitrarily controlling program behavior. CFI dictates that software
execution must follow a path of a control-flow graph that is
determined ahead of time by analysis (in this case, static binary
analysis is performed). CFI is enforced using static verification and
binary rewriting (with Microsoft's Vulcan \cite{vulcan} tool) that
instruments software with runtime checks. These checks aim to ensure
that control flow remains within a given control-flow-graph. CFI is a
very effective defense against buffer overflow attacks (and any attack
which attempts to change a program's control flow) since any attempt
by an attacker to divert the control flow of a program will be caught
by CFI. However, the main barrier to CFI's adoption seems to be the
overhead associated with the scheme. The average overhead of CFI in
the prototype implementation is 16\% on the SPEC2000 benchmarks. Also, unlike SecondWrite,
the binary rewriter used by CFI depends on a binary being compiled
with debug information which is usually not available in production
binaries. If a binary is not compiled with debug information then CFI
cannot be currently applied.

Our schemes implemented through our binary rewriter can provide the
same level of protection as CFI. An additional advantage of our scheme
is that our binary rewriter does not require access to any special
information in an input binary unlike all previous binary rewriters
(including the binary rewriter used in CFI) which require access to
relocation or debug information.

% Padraig: Please add a reference to the blank cite I added above for
% Microsoft's Vulcan tool.  DONE

\mypar{Program Shepherding} Program Shepherding \cite{shepherd}
employs an efficient dynamic software machine-code interpreter
(DynamoRIO \cite{drio}) for implementing a security enforcement
mechanism. A broad class of security policies can be implemented using
a machine interpreter such as DynamoRIO. For example, DynamoRIO could
be used to enforce control-flow integrity. Program shepherding
enforces a similar policy that imposes certain runtime restrictions on
control flow such that an attacker can not alter a program's flow of
control.

Program Shepherding can experience significant memory and runtime
overheads, particularly on the Windows platform. The scheme requires
an application and interpreter to be run simultaneously. The high
overheads of interpretation in some cases are likely to limit adoption
of Program Shepherding. Further, unlike using off-line rewriters like SecondWrite, Program Shepherding requires the installation of an extra piece of heavyweight software (DynamoRIO) in addition to the application to be run.

\subsection{Related Work in Binary Rewriting}

Binary rewriting and link time optimizers have been considered by a
number of researchers. Binary rewriting research is being carried out
in two directions: static rewriting and dynamic rewriting. Dynamic
binary rewriters rewrite the binary during its execution. Examples are
PIN \cite{pin}, BIRD \cite{bird}, DynInst \cite{dyninst}, DynamoRIO
\cite{drio} and Valgrind \cite{valg}. Dynamic rewriters are hobbled
since they do not have enough time to perform complex compiler
transformations; they have been primarily used for code
instrumentation and simple security checks in the past. Moreover
dynamic rewriters do not have the time to perform deep code analysis
needed to discover program features needed for static optimization of
security checks. Finally dynamic rewriters encounter run-time overhead
from the act of rewriting, which can be substantial. Given these
drawbacks, we do not discuss dynamic rewriters further.

The methods in this research are primarily directed at static binary
rewriters such as our rewriter, SecondWrite. Existing static binary
rewriters include Etch \cite{etch}, ATOM \cite{atom}, PLTO
\cite{plto}, Diablo \cite{Diablo1}, and Vulcan \cite{vulcan}. Three
points of novelty for our work are as follows. First, we are not aware
of any rewriter adding our particular set of existing compile-time
security schemes to binaries. Second, none of the existing rewriters
employ a compiler level intermediate representation; rather they
define their own low-level machine-code-like custom intermediate
representation. This has several downsides: $(i)$ most existing
rewriters cannot modify the stack layout since they do not distinguish
individual objects on the stack. Hence they cannot implement security
schemes that modify the stack; and $(ii)$ most existing rewriters
recognize functions, but not their arguments or return values, and
hence cannot deploy security schemes that employ these schemes. Our
Secondwrite rewriter overcomes both these problems; due to space
limitations, the pertinent details can be found elsewhere
\cite{blind-kapil}.

A third point of novelty of our work is that all existing rewriters
can only rewrite binaries that contain relocation or debug
information.  This information, present at link-time, is usually
discarded in COTS binaries for two reasons -- it is not needed for
execution; and vendors legitimately fear it can be used to reverse
engineer their binaries.  Indeed of twenty commercial and open-source
binaries we surveyed, \emph{none contained either relocation or debug
  information}.  As a result, existing binary rewriters would not be
able to rewrite those binaries at all.  \underline{In effect, existing
  binary rewriters can only be deployed by developers, not end-users.}
In contrast our rewriter (Secondwrite) can rewrite arbitrary binaries
even without relocation or debug information, as we have described in
previous work \cite{blind-matt}.  This renders our platform a uniquely
powerful tool for allowing anyone to rewrite binaries from any source
to enable any security scheme they want.

%Diablo defines an augmented whole program control-flow-graph based intermediate representation with program registers as globals and memory as a black box. It does not attempt to obtain high-level information like function prototypes and is geared mainly towards  optimizations like code compaction. Taking memory as a black box limits its applicability to architectures like x86 which have a limited set of registers. ATOM defines a symbolic RTL-based intermediate representation with infinite registers but does not make any attempt of analyzing or modifying the stack layout. Its mainly targeted towards RISC architectures. PLTO employs a whole program CFG based IR and implements stack analysis to determine the use-kill depths of each function. However, this information is not used for converting it into high-level IR; rather it is only used for low-level custom optimizations like load/store forwarding. Etch does not explicitly build an intermediate representation build an intermediate representation and allows users to add new tools to analyze binaries. The primary goal of Etch appears to be instrumentation and has only been shown to be applicable for simple optimizations like profile-guided code layout.
%
%
\subsection{Summary}

Current defenses have a number of weaknesses:

\begin{itemize}

\item They are not easily deployable

  \begin{itemize}
  \item Source code for an application is required
  \item An application needs to be compiled with certain information
  \item Hardware support is required
  \end{itemize}

\item Most are not readily available for use and evaluation

\item Some suffer from un-acceptable overheads

\item No scheme is immediately applicable to all binaries

\end{itemize}

Using the novel binary rewriting techniques developed within our research group, we have developed a scheme with the following characteristics:

\begin{itemize}

\item It is immediately deployable

\item It is applicable to any binary

\item Access to an application's source code is not required

\item An application does not need to be compiled in any way such that it contains special information e.g. debug or relocation information

\item It has low overheads

\item Many common stack-based buffer overflow attacks are prevented

\end{itemize}

%
% Padraig: I deleted the summary above and moved it to a file since I
% thought it was overly simplistic and unnecessary.  DONE - not really
% anything to do here...
