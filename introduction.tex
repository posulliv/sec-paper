\section{Introduction}
\label{sec:introduction}

% Padraig: I just read the call for papers for ACSAC and it says the
% papers should be blinded.  Please read:
% http://www.acsac.org/2010/cfp/papers/ As a result, please remove the
% author names and all self-references.  Right now there are two I
% added: Matt's patent and kapil's pact paper.  Please replace those
% with new citations saying only "Suppressed for blind review."  You
% can leave the Secondwrite name there since no one yet knows where it
% comes from.  DONE

Despite considerable research and work on programmer education and
tools, programming language and compiler support for security,
hardware and operating system features, low-level software
vulnerabilities remain an important source of compromises and a
perennial threat to system security. While other sources of
vulnerability have emerged more recently, such as SQL injection,
cross-site scripting (XSS) and cross-site request forgery (XSRF),
binary-level vulnerabilities continue to be discovered in very popular
software \cite{flash-overflow,ms-overflows} and to be exploited for
fun and profit \cite{buffer-overflows}.

The lack of convergence to a comprehensive solution can be attributed
to several factors, consisting of a mix of the technical and
non-technical. At the core, there exists a fundamental dichotomy in
the capabilities and motivation of producers and consumers of
software, vendors and end-users/administrators respectively. On the
one hand, software producers are probably in the best position to both
proactively and reactively prevent and mitigate such vulnerabilities:
they have access to the source code, the compiler tool chain, and the
developers themselves. As a result, they can apply security mechanisms
that offer high coverage and effectiveness at low overhead, because
they are applied at the point where the most semantic knowledge about
the program and the code is available. On the other hand, it is
software consumers that face the risk and bear the costs of compromise
due to software vulnerabilities and are the most motivated to take
action, often localized, to mitigate a newly discovered
vulnerability. However, consumers often only have access to the
program binary and configuration files. Thus, absent vendor patches
(which can often take a long time and may contain bugs \cite{patches})
consumers can only use security mechanisms that treat the
software as a black box. Inevitably, such mechanisms resort to
isolation ({\it e.g.,} through a virtual machine) or to behavioral
detection ({\it e.g.,} system call monitoring), with attendant costs,
complexity and risk. Even security-conscious software consumers often
cannot properly evaluate the risks they face because they do not know
what security mechanisms, if any, a producer has used in their
development process and tool-chain \cite{securitybugs2003}.

% Padraig: Please fix all the XXX comments Angelos inserted above
% before returning the document to him for final review.  DONE -
% wasn't sure about the paper related to quality of patches but the
% paper I reference makes some compelling points that I thought made
% the same argument on the quality of patches. It also makes the point
% that patches for commercial products can take quite long to be
% released

We present a new mechanism based on advanced binary rewriting that
seeks to bridge the gap between incentive/motivation and capabilities
on the consumer side. Our approach allows end users to retrofit
powerful security mechanisms into third-party, binary-only
software. These mechanisms are well-known, and some of them have been
{\it partially} integrated in separate tools and development
environments ({\it e.g.,} ProPolice in {\it gcc} and the optional {\it
  /GS} flag in Visual Studio). Our system allows end-users to ensure
that the software they run on their systems uses any and all such
features, regardless of the choices or capabilities of
vendors\footnote{It is worth pointing out that not all development
  tool-chains support a given security feature, while vendors and
  products are often intimately tied to them. As a result, there is
  considerable reluctance by vendors to switch to a ``better''
  compiler, for example, even if such existed.}. Furthermore, our
approach allows end-users to selectively apply different defense
mechanisms to different parts of the program, based on their own
analysis, risk assessment, and knowledge of potential or actual
vulnerabilities in the code. Essentially, we provide the same
self-defense capabilities that open-source software users {\it can}
utilize to users of binary-only software\footnote{Of course, just
  because open-source software users {\it can}, does not mean they
  generally {\it do} assess or modify/secure their installations.}.

The contributions of this paper are twofold. First, we present a
powerful binary-rewriting framework in the context of software
security. Specifically, we investigate the ability of such a system to
retrofit known invasive, powerful and low-overhead security mechanisms
to program binaries, in the absence of source code or even debugging
symbols. Second, we evaluate the effectiveness and efficiency of our
scheme and of the retrofitted security mechanisms, as compared to
other ways in which these and similar security mechanisms can be
applied to software. We conclude that a system such as ours would
enable software consumers to protect themselves at the same level of
effectiveness as if vendors had taken similar steps ({\it i.e.,} used
the same security techniques) and at equally low overhead. Thus, we
believe that we have removed a significant factor in improving the
overall security posture of systems against low-level software
compromises.

An additional contribution of this paper is that we have carefully
chosen a set of complementary and effective schemes that, taken
together, achieve the goal of defending against all types of buffer
overflow attacks at the lowest combined run-time cost.  The totality
of our schemes protect against buffers on the global, stack, and heap
segments from overflowing onto a variety of (usually code) pointer
locations that are vulnerable to attack, including return
addresses, function pointers, indirect branch pointers, \emph{longjmp}
buffers, and base pointers. This is an important practical
contribution in itself, as this is the first solution in the
literature to retrofit a comprehensive set of protections against
buffer overflow attacks, which are still very common, into arbitrary
new and legacy binaries.  We intend to make this tool available
publicly soon.

The remainder of this paper is organized as
follows. Section~\ref{sec:related} gives an overview of related work
in binary rewriting and binary-only software security
mechanisms. Section~\ref{sec:background} presents background on binary
rewriting, and how rewriting relates to security. We describe the
methods we have chosen in Section \ref{sec:methods}, and discuss
experimental results in Section~\ref{sec:results}. We conclude with
our thoughts for future work in Section~\ref{sec:conclusions}.
