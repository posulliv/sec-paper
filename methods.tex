\section{Methods}
\label{sec:methods}

One of the contributions of this paper is that we have carefully
chosen a set of complementary and effective schemes that, taken
together, achieve the goal of defending against all types of buffer
overflow attacks at the lowest combined run-time cost.  The totality
of our schemes protect against not only the commonly known stack
buffer overflow into return addresses, but is much more general than
that, in that they protect against buffers on the global, stack and
heap segments from overflowing onto a variety of code pointer
locations that are possible in any data segment, including return
addresses, function pointers, indirect branch pointers, \emph{longjmp}
buffers, and base pointers\footnote{Base pointers are not code
  pointers but lead to a similar vulnerability \cite{wilander2003}.}.

We implement our scheme by adding various passes that operate on
high-level IR inside our binary rewriter. Our overall scheme consists
of a number of components that we describe in detail in this section.

\subsection{Stack Canary Insertion}

The first component of our scheme is the simplest. LLVM provides the
ability to insert stack canaries during code generation. Utilizing
this capability allows us to easily provide the same level
of protection to an un-protected binary as
StackGuard~\cite{stackguard-98} would provide when given an
application's source code.

Essentially, a random canary value is generated at run-time and placed
on the stack during a function's prologue. In the function epilogue,
the value stored on the stack is compared with the random canary value
for this process. If there is any difference, execution is halted as
the canary value has been corrupted.

\subsection{Base Pointer Elimination}

The \emph{old base pointer} which resides on the stack is a data
pointer that points to the base of the parent function's
stack frame. Compilers sometimes introduce it since it makes it
convenient to restore the stack pointer at the end of the function and
to address different stack locations with the same offset even as the
stack grows and shrinks in the function. When it is present in the
input binary, it introduces a vulnerability just as dangerous as a
code pointer~\cite{wilander2003}. This is because the old base pointer
can be attacked by building a fake stack frame with a return address
pointing to attack code, followed by overflowing the buffer to
overwrite the old base pointer with the address of this fake stack
frame. Upon return, control will be passed to the fake stack frame
which immediately returns again redirecting flow of control to the
attack code.

Given our unique use of LLVM IR in Secondwrite, the elimination of the
base pointer in the output binary becomes a simple matter even when
the input binary has base pointers. LLVM is an optimizing compiler and
the binaries produced by LLVM are highly optimized. One common
optimization applied by modern compilers on the x86 platform is to
free up the EBP register for register allocation by removing the base
(or frame) pointer. We used this LLVM pass to eliminate the base
pointer from the binary.

When the base pointer is eliminated by LLVM, any attack relying on overwriting the base pointer
is immediately prevented. There will be no base pointer for an
attacker to modify. While corruption of the stack may still occur if
an attacker overflows a buffer in order to attempt to overwrite the
base pointer, no attack will be successful.

\subsection{Return Address Protection}

Given that stack canaries as inserted by LLVM do not provide the same
level of protection as the ProPolice mechanism that comes with GCC, we
decided to implement a more complete solution similar to the
protection scheme in StackShield \cite{stackshield}, that protects
against corruption of the return address. The basic idea of our return
address protection scheme is as follows:

\mybeginenumerate

 \item During the function prologue, push the return address of the
   current function in a return address stack implemented in a global
   variable.

 \item In function epilogue, compare the current return address on the
   stack with the value popped from the top of the return address
   stack.

 \item If there is any difference between these values, execution is
   halted.

\end{enumerate}

This simple scheme will detect if the return address has been modified
either directly or indirectly. We implemented this scheme as it is
relatively simple and protects against both direct and in-direct
modifications of the return address. It also requires no modification
of the stack layout and prevents modifications of the return address
through buffer overflows in the heap or global segments.

Two challenges with this scheme are as follows. First, its overhead
might be significant since every function has an associated security
overhead incurred every time it is called. We found this overhead to
be especially significant for recursive functions since they tend to
short-running. Second, the size of the return address stack might be
significant for deeply nested recursive functions, and we would have
to bound it a-priori, which is hard to do.

We applied an optimization for relieving this problem which we call
the \emph{return address check optimization}. We observed that this
protection mechanism is only necessary if a function contains a write
to a stack buffer since return addresses only exist on the stack. This
is hard to determine without symbolic information, so we
conservatively try to prove that a function only has directly
addressed memory references to constant addresses.  If it finds any
indexed write (base + runtime-variant offset), then it conservatively
assumes that it could be a buffer write, and disables the
optimization. If all writes are provably non-indexed writes to a
constant offset, it enables the optimization, \ie the protection
mechanism is turned off in the function. Thus the optimization saves
on run-time overhead without sacrificing any protection.

% Padraig: Please complete the above paragraph.  Talk about why you
% think this scheme is better than stack canaries and propolice.  DONE

% Padraig: In the results section we should present results on the
% impact of the return address check optimization.  NOT DONE - raw
% numbers have been collected but not sure how to show this and stay
% within the page limit required for the conference

We found this optimization surprisingly effective since it works best
for small leaf functions in the call graph, and for recursive
functions, which happen to be precisely the functions dynamically
called most frequently.  During our experimental evaluation of our
scheme, of the many recursive functions we found, every one of them
had its check optimized away.  This is unsurprising since recursive
functions tend to be short running, and unlikely to allocate stack
arrays (although they may access portions of global arrays, such as in
quicksort, but those still are optimized.)  As a result of the
optimization, the run-time overhead for scheme is greatly reduced, and
the required return address stack depth is also greatly reduced.  Of
course, the overflow of the return address stack is not an error as we
add extensions to it on the heap upon overflow, which slows execution,
but is extremely rarely invoked even for small return address stack
sizes of (say) 256 addresses.

\subsection{Function Pointer Protection}

One common attack method used by attackers is to overwrite a function
pointer so that when it is de-referenced, code of the attacker's
choosing will be executed. In a binary executable, function pointers
will appear as indirect calls. Thus, another component of our scheme
concentrates on protecting all indirect calls and branches similar to
how function pointers are protected in StackShield \cite{stackshield}.

Our scheme adds checking code before all indirect calls and
branches. A global variable is declared at the end of the data segment
and its address is used as a boundary value. The checks inserted
before any indirect call or branch ensure that the target of the
indirect call or branch points to memory below the address of the
global boundary variable. If the target points above the address of
this global boundary variable then execution is halted.

% Padraig: Need to explain why the above scheme works.  What invariant
% are you assuming about the way OSs lay out the global and code data
% segments?  You should allow the function pointer to point to the
% code segment, but not the data segments, and not dlls either.  How
% does the above scheme ensure that? Does the scheme work when the
% attacker changes the function pointer to point to the C library
% (return-to-libc attack?)  DONE

An assumption in the above scheme is that a process follows the
standard UNIX layout with the data segment above the code
segment. This scheme does not protect against return-to-libc attacks
since the target of the indirect call will still be within the code
segment.

\subsection{Protection for \emph{longjmp} buffers}

The paired functions \emph{setjmp} and \emph{longjmp}, present in most
C and C++ libraries, provide a means to alter a program's control flow
in addition to the usual subroutine call and return sequence. First,
\emph{setjmp} saves the environment of the calling function (say
\emph{foo()})into a data structure, and then \emph{longjmp} in another
function (say \emph{bar()}) can use this structure to jump back to the
point it was created, at the \emph{setjmp} call. As a result,
execution will return from \emph{bar()} to \emph{foo()} even when
\emph{foo()} is not the immediate parent of \emph{bar()}. A typical
use for \emph{setjmp}/\emph{longjmp} is exception handling.

The data structure used by \emph{setjmp} for saving the execution
state is referred to as a \emph{jmp\_buf}. Within this structure,
enough information is stored to restore a calling environment. In
particular, one member of this structure saves the value of the
program counter which is used when restoring the calling
environment. An attack method used by attackers is to overwrite the
value of the program counter stored in the \emph{jmp\_buf} structure
after a call to \emph{setjmp} and before a call to \emph{longjmp}. If
this happens, control will be transferred to an address of the
attacker's choosing when the \emph{longjmp} is executed. Our method
for defending against attacks of this kind is as follows:

\mybeginenumerate

 \item Create a hash table within the global segment of the rewritten
   binary.
 \item After each call to \emph{setjmp} store the current value of the
   program counter member of the \emph{jmp\_buf} structure in the hash
   table.
 \item Before a call to \emph{longjmp} get the current value of the
   \emph{jmp\_buf} structure that will be used. Attempt to perform a
   lookup in the hash table for the value of the program counter.
 \item If the lookup in the hash table fails, then the value of the
   program counter has been modified and so we abort; otherwise
   execution continues

\end{enumerate}

We expect the run-time overhead of this scheme to be very low in
practice, since \emph{setjmp} and \emph{longjmp} calls are very
rare. To the best of our knowledge, this scheme is the first
protection scheme designed to protect against longjmp buffer attacks
in the manner described.

% Padraig: You need to say here if this method is a new method of our
% creation, an existing method, or a variant of an existing method.
% In the later two cases, give a reference.  If it is new, say so and
% claim credit for it!  DONE
